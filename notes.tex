\documentclass{article}
\usepackage[utf8]{inputenc}

\title{IntelligentAgent}
\author{Liangwei CHEN }
\date{November 2019}

\usepackage{natbib}
\usepackage{graphicx}

\usepackage{amsmath} % for "\DeclareMathOperator" macro
\usepackage{amssymb} % for "\mathbb" macro
\begin{document}

\maketitle

\section {Real-world Games}
\begin{itemize}
    \item Mixed Nash Equilibrium may not be optimal for all participants implies Need for mediator for cooperation.
\begin{tabular}{llll}
          & Mediator      & cooperate & defect \\
Mediator     & (9, 9)  & (10, 0)   & (5, 5) \\
cooperate & (0, 10) & (9, 9)    & (0, 10) \\
defect    & (5, 5)  & (10, 0)   & (5, 5)
\end{tabular}

    \item Equilibrium types: 
    \begin{itemize}
        \item For simplicity we denotes $\sigma$ as the equilibrium strategy distribution in all following expression
        \item Pure Nash Equilibrium: $$s = \sigma,\; \forall i \in AGENTS, \forall s^{'} \in S_{i}, \; C_{i}(s) \leq C_{i}(s^{'}, s_{-i})$$ 
        Properties: May not exist, Best Price of Anarchy bound.
        
        \item Mixed Nash Equilibrium: $$\forall i \in AGENTS, \forall s^{'} \in S_{i}, \; \mathbb{E}_{s\sim \sigma}[C_{i}(s)] \leq \mathbb{E}_{s\sim\sigma}[C_{i}(s^{'}, s_{-i})]$$
        Assume INDEPENDENT strategy decision. Existence promise. Hard to find.
        
        \item Correlated Equilibrium: 
        $$\forall i \in AGENTS,\; \forall s_{i} \in S_{i}\; s.t.\; \mathbb{P}_{\sigma}(s_{i})>0,$$
        $$\mathbb{E}_{s_{-i}\sim \sigma | s_{i}}[C_{i}(s_{i}, s_{-i}) | s_{i}]
        \leq \mathbb{E}_{s_{-i} \sim \sigma | s_{i}}[C_{i}(s^{'}, s_{-i}) | s_{i}]$$
        Interpretation: For any agent and any non-trivial deterministic strategy of this agent, if other agents choose their strategy according to correlated equilibrium distribution under the condition of this agent's strategy, then moving to other deterministic strategy cannot reduce the cost of this agent. \newline
        The fact that MNE is CE can be seen by analyzing the optimality of each non-trivial deterministic strategy. The only difference between MNE and CE is that CE allows CORRELATED strategy distribution.
        
        \item Coarse Correlated Equilibrium:
        $$\forall i \in AGENTS,\; \mathbb{E}_{s \sim \sigma}[C_{i}(s_{i}, s_{-i})] \leq \mathbb{E}_{s \sim \sigma}[C_{i}(s^{'}, s_{-i})]$$
        The only difference between CE and CCE is that CE promises optimality under every condition while CCE only promise optimality when consider all non-trivial strategies together.
        
        \item Relationship:
        $$PNE \subset MNE \subset CE \subset CCE$$
    \end{itemize}
    
    \item Negotiation
    \begin{itemize}
        \item Alternating negotiation without discount factor: Agents propose in turn while the resource's value does not decrease with time. Only the last proposal matters.
        \item Alternating negotiation with discount factor: Agents propose in turn while the resource's value decrease with time. Solve the trivial-but-last problem. Results depend on who start first.
        \item Parallel negotiation: Agents exchange proposals in parallel. Solve the order problems above. \newline
        Goal: Nash Bargaining Solution: Find the deal $D^{*}$ such that:
        $$D^{*}=argmax_{D}(\Pi_{i \in AGENTS}r_{i}(D))$$
        Namely find the deal maximize the product of rewards. \newline
        Algorithm: Monotonic concession protocol (2-agents). \newline
        Since the reward function $r$ is private to every agent, the agent needs to calculate a number containing its response to both proposals locally. These numbers should then be exchanged and indicate which of the proposal leads to bigger product of reward. The agents will then agree to move in that direction. \newline
        Assume two agents $i$, $j$ in the system. Agent $i$ calculates $f_{i}=1 - \frac{r_{i}(D_{j}) - r_{i}(D_{conflict})}{r_{i}(D_{i}) - r_{i}(D_{conflict})}$. Similarly for agent $j$. \newline
        Notice that $f_{i} > 1\; iff\; r_{i}(D_{j}) < r_{i}(D_{conflict})$, in that case agent $i$ would rather stop the negotiation and choose the conflicting plan directly, for instance in the scenario of wireless transmitting, it will attempt to transmit all the time regardless of what its peer does. \newline
        Furthermore, $f_{i} > f_{j}$ indicates
        $$(r_{i}(D_{i}) - r_{i}(D_{conflict}))(r_{j}(D_{i}) - r_{j}(D_{conflict})) >$$ $$ (r_{i}(D_j) - r_{i}(D_{conflict}))(r_{j}(D_{j}) - r_{j}(D_{conflict}))$$.
        Thus the agent $j$ will know he should adjust the plan to move in the direction of $i$'s proposal.
        
        
    \end{itemize}
\end{itemize}
\end{document}
